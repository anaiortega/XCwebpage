


%\section{Introducción y objeto}
Formando parte de las \emph{Infraestructuras Generales de Saneamiento y Depuración de la Cuenca de Arroyo de la Reguera}, se realiza la hinca de un colector de aguas residuales de sección circular de 1800 mm de diámetro. Para ello es necesario diseñar cuatro pozos de ataque, de 12 metros de diámetro y altura libre entre 14 y 17 metros. 
Como premisas de diseño, se establecen las siguientes:
\begin{itemize}
\item El empuje ejercido por los gatos se prevé que sea de $12000$ kN.
\item La solución adoptada para la contención de tierras ha de ser estanca, de manera que pueda trabajarse en el pozo sin necesidad de agotamiento.
\item En la pantalla no pueden disponerse anclajes activos ni pasivos.
\item El ámbito interior de la excavación debe quedar completamente diáfano, de manera que no se entorpezca la instalación y retirada de la maquinaria de empuje.
  \item El suelo, de acuerdo con los resultados del informe geotécnico encargado por la empresa constructora, tiene un ángulo de  rozamiento interno muy bajo (en general en torno a los $19^o$ alcanzando en algunos ensayos $13^o$).
\end{itemize}



Con estos condicionantes, se estudió la idoneidad de las siguientes tres alternativas:
\begin{enumerate}
\item Pantalla de pilotes secantes de directriz circular.
\item Pantalla de paneles de hormigón armado de directriz poligonal (octogonal o similar).
\item Pantalla de tablestacas de directriz circular.
\end{enumerate}
 La solución de pantalla de paneles de hormigón armado quedó descartada debido a la dificultad de conseguir su estanqueidad con una directriz poligonal. Por su parte, la constructora puso objeción a la solución de tablestacas por observar dificultades para su hinca. En consecuencia se dimensionó la primera de las soluciones planteadas.



Se diseñó en primer término una pantalla de pilotes secantes de 55 cm de diámetro, la cual presentaba un excelente comportamiento frente a los empujes de tierras, resultando una armadura mínima para los pilotes secundarios. Para evitar el posible sifonamiento en el fondo de la excavación se propuso la ejecución, previamente al inicio de la excavación, de un tapón de \emph{jet grouting}. El reparto de las cargas transmitidas por los gatos entre todos los pilotes del perímetro se realizaba a través de un muro de  un macizo interior, formado por una losa inferior de 1.5 metros de espesor y un muro perimetral de 4.5 metros de altura y 1.1 metros de espesor. 



Tras el inicio de la obra, se observaron una serie de dificultades para la ejecución de la solución estructural diseñada.  La coyuntura del sector de la construcción, que presentaba una gran actividad en aquel momento,  hizo que fuera imposible disponer de los medios adecuados para la ejecución de la pantalla de pilotes secantes.  En paralelo se realizó una nueva investigación geotécnica que arrojó mejores resultados del ángulo de rozamiento interno del suelo, lo que permitió construir la pantalla discontinua, formada por pilotes separados 5 cm entre sus caras, que se perforaban con barrena continua.  Para su ejecución, de acuerdo a los resultados del cálculo, se hizo necesario adoptar las siguientes medidas:

\begin{itemize}
\item Garantizar, mediante agotamiento, el rebajamiento del nivel freático por debajo  de la cota de excavación.
\item Disponer varios niveles de codales a lo largo de la altura de excavación.
\item Controlar desplazamientos de la pantalla durante todo el periodo de servicio de la misma.
\item Evitar sobrecargas en el trasdós de la pantalla.
\item Acelerar la ejecución de las últimas fases de excavación y del macizo de anclaje para que el terreno no desarrollara la totalidad del empuje.
\end{itemize}

