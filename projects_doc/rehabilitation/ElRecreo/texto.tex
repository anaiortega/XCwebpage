

El deficiente estado del forjado que sustentaba la plaza de la urbanización El Recreo, sita en el municipio de Móstoles, así como la necesidad de posibilitar el acceso a dicho recinto de vehículos de emergencia, en particular camiones de bomberos, hizo que desde el Ayuntamiento de esta localidad se plantease la solución de este problema mediante el refuerzo de este forjado de modo que sea capaz de soportar las sobrecargas inducidas por tal servidumbre.

Con el diseño del refuerzo se pretendía atender dos objetivos fundamentales:

-Dotar al forjado de resistencia suficiente para resistir las acciones prescritas por la normativa vigente.
- Minimizar la afección de las obras a los establecimientos situados bajo la plaza y reducir al mínimo posible las molestias a los vecinos durante las obras.

Se planteó una primera solución, que consistía básicamente en la construcción, sobre la estructura principal existente, de un nuevo forjado formado por una losa reticular mixta que se apoya exclusivamente sobre las cabezas de los pilares. La losa reticular mixta era una estructura autoportante, formada por mallas de acero superior e inferior conectadas por diagonales, las cuales forman una pirámide de base cuadrada. Como elemento de rigidización, que además forma la superficie del forjado, se disponía una losa superior de hormigón de reducido espesor.

Con solución se lograba resistir una sobrecarga de uso de cierta entidad ($> 20 kN/m^2$) con una estructura de muy bajo peso propio y relativamente poco canto. Además el empleo de este tipo de estructura permitía introducir sobre la misma cargas moderadas en fases tempranas de su ejecución, lo que facilitaba establecer pasillos para el tránsito de peatones durante las obras. 

La secuencia de ejecución se planteó en las siguientes fases:

- Apeo del forjado existente, refuerzo de pilares (sólo en caso necesario), y picado del paquete de acabado superficial (baldosas y mortero).
- Repicado y limpieza de las cabezas de pilares y colocación de la malla metálica espacial.
- Colocación de encofrado para el posterior hormigonado de la losa superior.
- Hormigonado de la losa superior.


Dicha secuencia de ejecución se realizaba por parcelas dentro de la zona de actuación de manera que el equipo de trabajo, en su avance desde la entrada a la plaza desde la calle hasta llegar a la linde con el jardín interior, fuera dejando detrás el forjado terminado sobre el que se podían transportar los materiales y equipos necesarios al frente de la obra. 


%-----------------
Al iniciarse las obras de rehabilitación  se presentaron las siguientes contingencias:

-Se detectó que el colector de pluviales del patio interior estaba roto, con lo que la solución prevista en proyecto para el drenaje del nuevo forjado, que consistía en desaguar buena parte la superficie hacia este colector, resulta inviable.
- Por otra parte, tras consultar a diversos talleres de prefabricación de armaduras, se llegó a la conclusión de que, debido a la elevada carga de trabajo que dichos talleres tenían en aquel momento (dada la particular coyuntura del mercado de la construcción en dicha época), el coste de elaboración de la estructura metálica prevista en proyecto resultaba superior al de otras alternativas que se desecharon inicialmente por que implicaban un gasto de material sensiblemente mayor. 


Ante esta circunstancias se ha optó por introducir las siguientes modificaciones respecto al proyecto original:

- Eliminar el canalón de recogida de pluviales que estaba previsto instalar en el patio interior para recogida de pluviales y cambiar la red de drenaje de modo que las aguas viertan al colector situado en la calle Pintor Velázquez.
- Sustituir la estructura reticular mixta prevista inicialmente, por un forjado de chapa colaborante sobre vigas mixtas.



-----------------------
The poor state of the slab that supported the square of the urbanization El Recreo, located in the municipality of Móstoles, as well as the need to enable emergency vehicles, in particular fire trucks, to access to this area, led the city council to consider the solution of the problem by reinforcing this slab.

The design of the reinforcement was intended to meet two fundamental objectives:

-Give sufficient strength to the structure in order to withstand the loads prescribed by the standards.
- Minimize the affection of the works to the shopping areas located under the square and reduce as much as possible the annoyances to the neighbors.

We designed a solution that basically consisted of the construction of a new slab that rested exclusively on the heads of the columns. The slab was a self-supporting structure, formed by upper and lower steel meshes connected by diagonals, which form a square-shaped pyramid. As an element of stiffening, which also forms the surface of the square, an upper concrete slab of reduced thickness was provided.

By using this system, it was possible to withstand the required load of 20 kN/m2 with very low self-weight slab of relatively little height. In addition, the use of this type of structure allowed to introduce moderate loads in the early phases of its execution, which facilitated the establishment of corridors for pedestrian transit during construction.


At the beginning of the works, the following unforeseen events were encountered:

- It was detected that the storm drain in the inner courtyard was broken, so that the solution planned for the drainage of the new slab, which consisted of draining much of the surface towards this collector, was not feasible.
- On the other hand, after consulting several steel prefabrication workshops, it was concluded that, due to the high workload that they had at that time of expansion of the construction market, the cost of preparing the metal structure designed was higher than that of other alternatives which were initially discarded because of their substantially higher use of material.

In view of these circumstances, it was decided to introduce the following modifications with respect to the original project:

- Eliminate the rain gutter that was planned to be installed and change the drainage network so that the water flows into the collector located at Pintor Velázquez Street.
- Replace the initially planned composite floor truss system by a composite deck slab with steel profile sheeting supported on composite beams.
